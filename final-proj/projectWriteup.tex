\documentclass[12pt]{article}
\usepackage{graphicx}
\usepackage{titlesec}
\usepackage{xlop}
\usepackage{url}
\usepackage{subcaption}
\usepackage{geometry}
\graphicspath{ {./images/} }
\usepackage[fleqn]{amsmath}
\usepackage{tikz}
\usepackage{listings}
\usepackage{karnaugh-map}
\geometry{
    a4paper, 
    top=20mm,
    left=20mm,
}
\setlength{\parindent}{4em}
\setlength{\parskip}{1em}
\renewcommand{\baselinestretch}{1.5}

\title{Final Project: ES4 Spring 2021}
\date{5/11/2021}
\author{Ibrahima Barry, WIlly Lin \\ Zach Osman, James Eidson \\ ECE Tufts University}

\titleformat{\section}
    {\normalfont\Large\bfseries}{\thesection}{1em}{}[{\titlerule[1pt]}]

\lstdefinestyle{myListingStyle} 
    {
        basicstyle = \small\ttfamily,
        breaklines = true,
    }

\begin{document}

\maketitle

\begin{flushleft}

\section{Overview}

In this lab, a dual digit display decoder(dddd) was designed using VHDL and it's output was
displayed on a 7-segment display decoder (with two digit places). First, VHDL
code (device.vhd) to alternately flash two LEDs was made. The purpose which was to eventually
do a similar thing with the two digit slots of the 7-seg. display decoder. The
idea was to have both LEDs alternate with great enough frequency so they
appeared to be on simultaneously. . Then, vhdl code (dddd.vhd) to produce the output of
the one's place and ten's place on the 7-seg. was made. Finally, using a DIP
switch with 6 bits wired up, the values $[0, 2^6 - 1]$ were displayed on the 7-seg.  


\section{Technical Description and Design}
\textbf{Flashing LEDs:}\\
The HSOC built-in module with a 26-bit counter was used to drive two LEDs. By
changing the bit of the counter used to drive the LEDs the ON-OFF frequency of
the LEDs was changed. It turned out at using the 18th bit of the counter made it
so that the two LEDs seemed to be ON at the same time:  \\

at $48MHz/2^{20}$ => still seems to blinking \\
at $48MHz/2^{19}$ => still seems to blinking just a bit \\
at $48MHz/2^{18}$ => steady not blinking \\

\hfill \break

\textbf{Two-Digit Display:}\\
By understanding how to make the two LEDs look they're on simultaneously, two
NMOS transistors were used as "switches" turning on and off each digit.
Here is the schematic: \\

%%%%%%%%%%%%%%%%%%%%%%%%%%%%%%%%%%%%%%%%%%%%%

Then in the top module (device.vhd) the code to switch between which LED output
is on was written. The logic for the left and right digits from (dddd.vhd) were
used as inputs to a MUX with the select input being the msb of the counter
driver. Then based on whether or not that bit was on or off one 7-bit output was
(for the ones-place or tens-place) was outputted. Here is the block diagram for
this process: \\


A zip file of all the source code will be provided. 

%%%%%%%%%%%%%%%%%%%%%%%%%%%%%%%%%%%%%%%%%%%%% 

This is the final implementation on the breadboard:\\

\section{Results and Testing}

The behavior of the circuit was farily straightforward to test. One could
manually use the DIP switches to do all $2^6$ possible number combinations which
would take very long (but doable). 

Doing this for all possible digits I was sure the circuit functioned correctly.
Also a lab TA looked over my work and agreed it worked as intended.\\

%%%%%%%%%%%%%%%%%%%%%%%%%%%%%%%%%%%%%%%%%%%%%%%%%
% Explain the simuation waves
%%%%%%%%%%%%%%%%%%%%%%%%%%%%%%%%%%%%%%%%%%%%%%%%

\section{Debugging Log}

\begin{enumerate}
    \item A few LEDs on the 7-seg weren't lighting up or they were very dim.
    \begin{itemize} 
        \item Manually setting the LEDs on would work but not otherwise. 
        \item some possible causes included something wrong with VHDL code, the
wiring of the breadboard, and a hardware component not working (not likely). 
        \item Resolution: I needed to ground both GND pins of the Upduino. After
that the dim effect went away. 
        \item Lesson: Always check ground connections because even if other
parts of the circut make sense you can get unexpected results. 
    \end{itemize}
    \item problem: The transistors weren't acting as switches for the LED on the
7-seg. properly. 
    \begin{itemize}
        \item I knew there was a problem because both the digits on the 7-seg
would always have the same numerical value. Both digits were displaying the
value of the ones place. 
        \item possible causes included wiring of the transistor, confusion of
what is the gate, drain and source. 
        \item The issue was that while the transistors were wired correctly they
weren't acting as switches because the power pins of the 7-seg for both digits
were also connected. I.e the transistors weren't controlling the flow of current
in a meaningful way since both the digits were always on regardless of the what
the transistors were doing. 
        \item the lesson I learned was to really understand what each component
does and not just wire it up mindlessly. If I understood the role of the
transistors, I would have recognized that the LED's on the 7-seg do not need to
be connected to power while the source of the transistor was already connected
to power. 
    \end{itemize}
    \item problem: The LEDs weren't giving values as expected from the input
given by the DIP switches. 
        \begin{itemize}
            \item There was a problem because when I toggled the switches the
LEDs weren't producing the correct values. 
            \item Possible causes included: FPGA pin's not connected properly,
the logic in the VHDL code. 
            \item It turned out that the reset signal on my counter was not set
to any value. After making this change the outputs seen on the LEDs of the 7-seg
were correct. 
        \end{itemize}

\end{enumerate}

\section{Reflection} 

\begin{enumerate}
    \item What was the most valuable thing you learned, and why?\\
        - I learned how to debug my VHDL code in a way different to how one
debugs code in a language like Java or Python. Many times the code can compile
but the hardware implementation may not behave the way you want it.
    \item What skills or concepts are you still struggling with? What will you do to 
          learn or practice these?\\
        - I'm still struggling with how to write testbenches for very large
complicated code. To practice I will review the textbook and try writing
testbenches for other VHDL code I've written. 
    \item This assignment took about 72 hours (not including several hours
tyring to write a testbench) during all of the  days in which I
worked on it. 

\end{enumerate}

\section{Work Divison} 

\section{source code}
%Add link to the github


\end{flushleft}
\end{document}
